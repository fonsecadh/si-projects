\documentclass[11pt]{llncs}
\usepackage[T1]{fontenc}
\usepackage{lmodern}
\usepackage{parskip}
\usepackage[colorlinks=true,urlcolor=Blue,linkcolor=black,citecolor=black]{hyperref}
\usepackage{graphicx}
\usepackage[utf8]{inputenc}
\usepackage[english]{babel}
\usepackage{fancyhdr}
\usepackage{csquotes}
\usepackage{lastpage}
\usepackage{array}
\usepackage[toc, nonumberlist]{glossaries-extra}
\usepackage[backend=biber, citestyle=ieee]{biblatex}
\usepackage[nottoc,numbib]{tocbibind}
\usepackage[activate={true,nocompatibility},final,tracking=true,kerning=true,spacing=true,factor=1100,stretch=10,shrink=10]{microtype}
\usepackage[hmargin=2cm,top=4cm,headheight=65pt,footskip=65pt]{geometry}

\title{Heuristic Search Practices. Course 2020-2021. N-Queens with A* and Genetic Algorithms.}
\author{Hugo Fonseca Díaz}

% Document
\begin{document}
% Abstract
\begin{abstract}
This is the abstract.
% Keywords
\keywords{A* \and Genetic algorithms \and N-Queens problem 
    \and Heuristic Search.}
\end{abstract}
% Introduction
\section{Introduction}\label{introduction}
In this paper we will conduct a experimental research on two strategies for solving the N-Queens problem, these two being the A* algorithm and a genetic algorithm. We will take measurements using a varying number of board sizes and compare the results not only between the two approaches, but also finding out which heuristic is best for each configuration in the case of the A* algorithm. 
\section{N-Queens Problem}\label{nqueens_problem}
Description of the N-Queens Problem.
\section{A* Algorithm}\label{astar_alg}
Description of the A* Algorithm.
\section{Genetic Algorithms}\label{gen_algs}
Description of the Genetic Algorithms.
\section{Application of A* Algorithm to the N-Queens Problem}\label{astar_nqueens}
Application of A* Algorithm to the N-Queens Problem.
\section{Applications of Genetic Algorithms to the N-Queens Problem}\label{gen_nqueens}
Application of Genetic Algorithms to the N-Queens Problem.
\section{Experimental Research}\label{exp_research}
Experimental Research.
\subsection{Dataset}\label{dataset}
Dataset
\subsection{A* Results}\label{astar_results}
A* Results.
\subsection{GA Results}\label{gen_results}
Genetic Algorithm Results.
\subsection{Comparison of A* and GA results}\label{comparison_astar_gen}
Comparison of A* and GA results.
\section{Conclusions}\label{conclusions}
Conclusions.
\end{document}

