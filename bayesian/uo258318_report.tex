\documentclass[11pt]{article}
\usepackage[T1]{fontenc}
\usepackage{lmodern}
\usepackage{parskip}
\usepackage[colorlinks=true,urlcolor=Blue,linkcolor=black,citecolor=black]{hyperref}
\usepackage{graphicx}
\usepackage{amsmath}
\usepackage[utf8]{inputenc}
\usepackage[spanish]{babel}
\usepackage{fancyhdr}
\usepackage{csquotes}
\usepackage{lastpage}
\usepackage{array}
\usepackage{listings}
\usepackage{color}
\definecolor{dkgreen}{rgb}{0,0.6,0}
\definecolor{gray}{rgb}{0.5,0.5,0.5}
\definecolor{mauve}{rgb}{0.58,0,0.82}
\usepackage[affil-it]{authblk}
\usepackage[activate={true,nocompatibility},final,tracking=true,kerning=true,spacing=true,factor=1100,stretch=10,shrink=10]{microtype}
\usepackage[hmargin=2cm,top=4cm,headheight=65pt,footskip=65pt]{geometry}

% Documento
\begin{document}
% Título
\title{SI. Redes bayesianas. Práctica obligatoria.}
\author{Hugo Fonseca Díaz\\ \email{uo258318@uniovi.es}}
\affil{Escuela de Ingeniería Informática. Universidad de Oviedo.}
\maketitle
% Descripción de la red bayesiana
\section{Descripción de la red bayesiana}
En este documento se expone el modelo utilizado a la hora de diseñar la red bayesiana para el ejercicio de las prácticas obligatorias de este temario.
% Situación modelada
\subsection{Situación modelada}
La red en cuestión modela las probabilidades de que un seguro de coche pague el valor del vehículo o lo arregle en caso de que se produzca un siniestro.

Para ello, se han creado diez nodos que representan posibles factores a tener en cuenta en dicha situación. Esos nodos son la marca y modelo del vehículo, la antigüedad del mismo, el tipo de seguro, si el conductor es titular o no del seguro, de quién es la culpa del accidente, si se produjo alguna imprudencia (conducción bajo los efectos del alcohol, conducir sin carnet o sin puntos, exceso de velocidad exagerado, etc), si la otra parte está asegurada, si el importe de la avería es superior al valor actual del vehículo, y por último si el seguro se hace cargo del pagado del valor del vehículo o de las reparaciones del mismo, en caso de que fuera conveniente.
\subsection{Estadisticas y comportamiento de la red}
En el caso de las estadísticas, se han realizado gracias a la experiencia personal de familiares y amigos que trabajan en concesionarios y que conducen en su día a día. También de ellos se han obtenido las situaciones específicas en las que un seguro de coche se hace cargo de los gastos o de la reparación de un siniestro, lo que ha sido de especial utilidad a la hora de diseñar los nodos de la red.
\end{document}
